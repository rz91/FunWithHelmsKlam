\section{Modellierung mit Unity}

\subsection{Gelände}
\begin{figure}[h]
\centering
\includegraphics[width=0.95\linewidth]{Abbildungen/Unity/TerrainProgress}
\caption{Entwicklungsschritte des Geländes}
\label{fig:TerrainProgress}
\end{figure}

\begin{wrapfigure}[8]{r}{0.5\textwidth}
	\vspace{-20pt}
	\begin{center}
		\includegraphics[width=0.45\textwidth]{Abbildungen/Unity/TerrainTool}
	\end{center}
	\caption{Terrain Werkzeuge}
	\label{fig:TerrainWerkzeuge}
\end{wrapfigure}

Die Abb. \ref{fig:TerrainProgress} zeigt verschieden Zwischenstände in der Gestaltung des Geländes. Die genutzten Werkzeuge sind in Abb. \ref{fig:TerrainWerkzeuge}, namentlich \textit{Raise/Lower} (1.), \textit{Paint Height} (2.) und \textit{Smooth Height} (3.), zu sehen. In allen drei Werkzeugen stehen verschieden Einstellungen für die Weite und Deckkraft des Pinsels zur Verfügung. Das Werkzeug \textit{Raise/Lower} ist besonders empfindlich und deshalb sollten für dieses nur niedrige Deckkraftwerte eingestellt werden.

%\newpage
Um ein gutes Größenverhältnis zwischen Burg und Gelände zu erreichen wurden die Maße des Terrains auf 500*500*600 festgelegt. Die Starthöhe des Geländes wurde mittels \textit{Paint Height} und \textit{Flatten} auf 50m gesetzt und von diesem Punkt heraus wurden die anderen Teile Stück für Stück herausgearbeitet. Mittel der Wahl war eine Kombination aus \textit{Paint Height} und \textit{Smooth Height}, gut zu sehen links oben in Abb. \ref{fig:TerrainProgress}. Zuerst wurde terrassenförmig die Höhe herausgearbeitet und danach wieder geglättet, um für fließende Übergänge zu sorgen. Das Gebirge im Hintergrund wurde größtenteils mit \textit{Raise/Lower} gestaltet. Es unterlag allerdings im Designprozess vielen Änderungen, da die Wirkung aus der Ego-Perspektive und das Zusammenspiel mit der Festung nicht optimal war. Im unteren Teil der Abb. \ref{fig:TerrainProgress} ist der finale Zustand des Geländes zu sehen.

\subsection{Texturierung}
\begin{figure}[h]
	\centering
	\includegraphics[width=0.95\linewidth]{Abbildungen/Unity/Texture}
	\caption{Wasserebenen und exemplarische Einstellungen}
	\label{fig:Textures}
\end{figure}

Das Gelände mit Texturen zu versehen ist in Unity denkbar einfach und selbsterklärend. Die Abb. \ref{fig:Textures} zeigt das Werkzeug zum Bemalen des Geländes. Es ist vergleichbar mit gängigen Grafikprogrammen. Über die Schaltfläche \textit{Edit Textures} können neue Texturen hinzugefügt werden. Danach stehen diese zur Verfügung und werden mithilfe des Pinsels auf das Gelände aufgetragen. Links in der Abbildung sind die weichen Übergänge von Sand, Erde, Gras und Fels sehen. Diese werden durch geringe Deckkrafteinstellungen und mehrmaliges Auftragen der Texturen erreicht. Eher schroffe Übergänge, unten rechts zu sehen, können durch höhere Deckkraft- und Stärkewerte (\textit{Target Strength}) erzielt werden. Die verwendeten Texturen stammen aus dem Assetstore.

\subsection{Wasser und Wasserfälle}
\subsubsection{Wasser}
In der Szene werden für die Darstellung von Flüssen und Ozean drei Wasserebenen benutzt. Als Asset wird das Standardwasser \textit{WaterProTime} aus Unity verwendet. In Abb. \ref{fig:Water} sind die drei Ebenen (1. kleiner Fluss, 2. großer Fluss und 3. Ozean) zu sehen. Zur Visualisierung des Gefälles wurden die Wasserebenen der Flüsse, im Bild rechts für Ebene 2. zu sehen, geneigt. Schwierigkeiten bei der Verwendung mehrerer Wasserebenen ergeben sich beim Übergang zwischen Zweien und der Wechselwirkung mit dem Gelände. Die Übergänge können beispielsweise durch die Verwendung von Wasserfällen kaschiert werden. 

\begin{figure}[h]
	\centering
	\includegraphics[width=0.95\linewidth]{Abbildungen/Unity/Water}
	\caption{Wasserebenen und exemplarische Einstellungen}
	\label{fig:Water}
\end{figure}

\subsubsection{Wasserfälle}
\begin{wrapfigure}{r}{0.5\textwidth}
	%\vspace{-20pt}
	\begin{center}
		\includegraphics[width=0.45\textwidth]{Abbildungen/Unity/Waterfall}
	\end{center}
	\caption{Wasserfall}
	\label{fig:Waterfall}
\end{wrapfigure}

Trotz der Neigung der Flüsse konnten die Höhenunterschiede zwischen den Wasserebenen nicht ausgeglichen werden. Dadurch wurde es notwendig Wasserfälle in die Szene einzufügen. Diese stammen aus dem Assetstore und wurden für die Zwecke des Projektes angepasst. Die Abb. \ref{fig:Waterfall} zeigt den Übergang zwischen dem kleinen und großen Fluss. Um den harten Wechsel von Wasserebene zu Wasserfall abzuschwächen, war es nötig eine kleine abgeschrägte Ebene passgenau einzufügen. Ebenso mussten die seitlichen Ränder mit dem Terrainwerkzeug verdeckt werden. Diese Technik wurde auch bei den beiden anderen Wasserfällen vom großen Fluss zum Ozean angewendet.

\section{Animationen}
\subsection{Unity Animationen}
\subsection{3d Studio Max Animation}
\subsection{Animation Controller}






 

